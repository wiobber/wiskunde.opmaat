\documentclass{ximera}
\input{../preamble}
\addPrintStyle{..}
\begin{document}
	\author{Wim Obbels}
	\xmtitle{De complex toegevoegde van een complex getal}{}
	\label{xim:complexe_getallen_complex_toegevoegde}

\begin{xmuitweiding}[Het tegengestelde van een reëel getal]
Het tegengestelde $-a$ van een reëel getal $a$ is een erg eenvoudig begrip, want $-a$ is 'het getal $a$ met het omgekeerde teken'.
Op de getallenrechte kan je $-a$ beschouwen als het spiegelbeeld van $a$ rond de oorsprong.

Het tegengestelde $-z$ van een complex getal kan onmiddellijk en zonder enig probleem worden gedefinieerd:als $z=a+bi$, dan is $-z = -(a+bi) = -a -bi$, en in het complexe vlak is $-z$ het spiegelbeeld van $z$ door de oorsprong:

\begin{image}[0.5\textwidth]
	\begin{tikzpicture}[scale=5]%,cap=round,transform canvas={scale=0.5}]
	
	\tikzmath{\hoek = 20;}
	
	\draw[->] (-1.2,0) -- (1.2,0) node[above] {Re$(z)$};
	\draw[->] (0,-0.5) -- (0,0.5) node[below right] {Im$(z)$};
	
	\draw[color=blue,thick] (0:0)  --  (\hoek:1); 
	\draw[color=blue,dashed] (0:0)  -- ({\hoek+180}:1); 

%	%
	\draw[color=black] (\hoek:1) node[name=P,circle, fill=black, radius=1pt,scale=0.8] {} node [yshift=1pt,above] {$z=a+bi$} ;  
%	%
	\draw[color=black] (\hoek:-1) node[name=Pt,circle, fill=black, radius=1pt,scale=0.8] {} node [yshift=1pt,below] {$-z=-a-bi$} ;  
%    %
%	\draw[dashed] (Pt) node[circle, fill=black, radius=1pt,scale=0.5] {} node[below] {$a$} -- (P);
	\draw[dashed] (Pt) -- (P);
	\draw[dashed] ({cos(\hoek)},0) node[circle, fill=black, radius=1pt,scale=0.5] {} node[below] {$a$} -- (P);
	\draw[dashed] ({-cos(\hoek)},0) node[circle, fill=black, radius=1pt,scale=0.5] {} node[below] {$-a$} -- (Pt);
	\draw[dashed] (0,{sin(\hoek)}) node[circle, fill=black, radius=1pt,scale=0.5] {} node[left] {$b$} -- (P);
	\draw[dashed] (0,{-sin(\hoek)}) node[circle, fill=black, radius=1pt,scale=0.5] {} node[left] {$-b$} -- (Pt);
	
	\end{tikzpicture}
\end{image}

\end{xmuitweiding}

Complexe getallen kan je spiegelen rond de $x$-as, en dat zal blijken een bijzonder nuttige operatie te zijn, die geen equivalent heeft voor een reëel getal. 
Terwijl men bij spiegelen door de oorsprong spreekt over het \textit{tegengestelde} $-z$, noemt men de over de $x$-as gespiegelde het \textbf{complex toegevoegde} $\overline{z}$. % of soms ook $z^*$.


\begin{definition}[Complex toegevoegde van een complex getal] \nl 
	
De \textbf{complex toegevoegde} van een complex getal $z=a+bi$, genoteerd $\overline{z}$, is het \textit{complexe} getal 
%$$
    \formulevb{\overline{z} \; \perdef \; \overline{a+bi} \; \perdef \; a-bi}{\overline{3+4i} = 3 - 4i.}
%$$
We noemen $z$ en $\overline z$ \textbf{complex toegevoegd} (aan elkaar), en beide zijn elkaars spiegeling over de $x$-as.
%
\begin{image}[0.4\textwidth]
	\begin{tikzpicture}[scale=4]%,cap=round,transform canvas={scale=0.5}]
	
	\tikzmath{\hoek = 20;}
	
	\draw[->] (-0.2,0) -- (1.2,0) node[above] {$\Re(z)$};
	\draw[->] (0,-0.4) -- (0,0.6) node[below right] {$\Im(z)$};
	
	\draw[color=blue,thick] (0:0)  --  (\hoek:1); 
	\draw[color=red ,dashed] (0:0)  -- (-\hoek:1); 
	%
	\draw[color=blue] (\hoek:1) node[name=P,circle, fill, radius=1pt,scale=0.5] {} node [below right] {$z=a+bi$};
	%
	\draw[color=red] (-\hoek:1) node[name=Pt,circle, fill, radius=1pt,scale=0.5] {} node [below right] {$\overline{z}=\overline{a+bi}=a-bi$};
    %
	\draw[dashed] (Pt) -- (P);
	\draw[dashed] (0,{sin(\hoek)}) node[circle, fill=black, radius=1pt,scale=0.3] {} node[left] {$b$} -- (P);
	\draw[dashed] (0,{-sin(\hoek)}) node[circle, fill=black, radius=1pt,scale=0.3] {} node[left] {$-b$} -- (Pt);
	
	\end{tikzpicture}
\end{image}

\end{definition}
 
\begin{example}\nl
	\begin{xmmulticols}[3]
        \begin{question}$\overline{3-4i}   = \answer[onlineshowanswerbutton]{3+4i}$ \end{question}
        \begin{question}$\overline{3+4i}   = \answer[onlineshowanswerbutton]{3-4i}$ \end{question}
        \begin{question}$\overline{-3-4i}   = \answer[onlineshowanswerbutton]{-3+4i}$ \end{question}
        \begin{question}$\overline{-3}   = \answer[onlineshowanswerbutton]{-3}$ \end{question}
        \begin{question}$\overline{i}   = \answer[onlineshowanswerbutton]{-i}$ \end{question}
        \begin{question}$\overline{-i}   = \answer[onlineshowanswerbutton]{i}$ \end{question}
	\end{xmmulticols}	
\end{example}

%Complex toevoegen werkt erg goed samen met optelling en vermenigvuldiging:

Volgende eigenschap, waarvan het bewijs een oefening is, toont dat complex toevoegen goed samenwerkt met optellen en vermenigvuldigen.


\begin{proposition}[Eigenschappen van complex toevoegen]\label{eig:complex_toegevoegde_som_product}
	Voor complexe getallen $z$ en $w$ geldt 
	\formulevb{\overline{z+w} = \overline{z} + \overline{w}}{\overline{(2+3i)+(4+5i)} = 6-8i = \overline{2+3i} + \overline{4+5i}}
	\vspace{-3mm}

	\formulevb{\overline{z\cdot w} = \overline{z} \cdot \overline{w}}{\overline{(2+3i)\cdot(4+5i)} = -7-22i  = \overline{2+3i} \cdot \overline{4+5i}}
\end{proposition}

\onlyOnline{
\begin{exercise} Bewijs de vorige eigenschap.
\end{exercise}
}

\begin{exercise} Bereken voor een complex getal  $z=a+bi \in\C$ volgende uitdrukkingen: 
	\begin{xmmulticols}[3]
	\begin{question} $z + \overline{z} = \answer[onlineshowanswerbutton]{2a}$
		\begin{oplossing}\nl
			$ z+ \overline{z} = (a+bi) + (a-bi) = 2a $
		\end{oplossing}
	\end{question}
\begin{question} $z -\overline{z} = \answer[onlineshowanswerbutton]{2bi}$
	\begin{oplossing}\nl
		$ z- \overline{z} = (a+bi) - (a-bi) = 2bi $
	\end{oplossing}
\end{question}
	\begin{question} $z\cdot\overline{z} = \answer[onlineshowanswerbutton]{a^2+b^2}$
		\begin{oplossing}\nl
			$ z\cdot\overline{z} = (a+bi)(a-bi) = a^2- i^2b^2 = a^2+b^2$
		\end{oplossing}
	\end{question}
	% \begin{question} $|z|^2 = \answer[onlineshowanswerbutton]{a^2+b^2}$
	% 	\begin{oplossing}
	% 		$ |z|^2= (\sqrt{a^2+b^2})^2 = a^2+b^2$
	% 	\end{oplossing}
	% \end{question}
%	\begin{question} $|z^2| = \answer[onlineshowanswerbutton]{a^2+b^2}$
%		\begin{oplossing}
%			$ |z^2| =  |(a+bi)^2| = | a^2+2abi -b^2)| = |(a^2-b^2) +2abi| = \sqrt{(a^2-b^2)^2 + 4a^2b^2} = \sqrt{(a^2+b^2)^2}= a^2+b^2$ 
%		\end{oplossing}
%	\end{question}
\end{xmmulticols}
\end{exercise}


Hiermee zijn volgende eigenschappen aangetoond:

\begin{proposition}[Eigenschappen van complex toevoegen]\label{eig:complex_toegevoegde}
Voor een complex getal $z=a+bi$ geldt 

\formulevb{\overline{\overline{z}} = \overline{(\,\overline{z}\,)} = z}{\overline{(\overline{2+3i})} = \overline{2 - 3i}  = 2+3i}
\vspace{-3mm}

\formulevb{z + \overline{z} = 2\Re(z)}{2+3i\;+\; \overline{2+3i} = 2 + 3i + 2 - 3i = 4 = 2\Re(2+3i)}
\vspace{-3mm}

\formulevb{z - \overline{z} = 2\Im(z)}{2+3i\;-\; \overline{2+3i} = 2 + 3i - 2 + 3i = 6i = 2i\Im(2+3i)}
\vspace{-3mm}

\formulevb{z\overline{z} = |z|^2}{(2+3i)(2-3i) = 4 - 9i^2 = 4 +9 = 13 = |2+3i|^2} 
  %\Ten  &  \frac{z}{\overline{z}} & = & \frac{z^2}{|z|^2} 
\end{proposition}

Zowel de som als het product van een getal $z$ en zijn complex toegevoegde $\overline{z}$ zijn steeds \textit{reëel}, en dat zal een manier opleveren om het inverse $z^{-1}$ te berekenen, en dus het quotiënt van twee complexe getallen.

\end{document}

