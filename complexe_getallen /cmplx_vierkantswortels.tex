\documentclass{ximera}
\input{../preamble}
\addPrintStyle{..}

\begin{document}
	\author{Wim Obbels}
	\xmtitle{Vierkantswortels van complexe getallen}{}
	\label{xim:cmplx_vierkantswortels}

    Een belangrijke motivatie voor het invoeren van complexe getallen was het creëren van een 'vierkantswortel uit $-1$', die we $i$ noteerden.
    Het is nu tijd om niet alleen de vierkantswortel van $-1$ te bestuderen, maar die van willekeurige complexe getallen.


    \begin{expandable}{remark}{Vierkantswortels van reële getallen}
        Je bent al lang vertrouwd met vierkantswortels van reële getallen. We herhalen hier kort de belangrijkste eigenschappen.

        \begin{definition}[Vierkantswortel van een reëel getal]\nl

            Een reëel getal $w$ is een {vierkantswortel} van een reëel getal $a$ als 
            \formulevb{a = w^2}{2 \text{ en } -2 \text{ zijn wortels van} 4 \text{ want } 2^2 = 4 \text{ en } (-2)^2 = 4}
        \end{definition}

        Enkel positieve getallen hebben vierkantswortels, en wel steeds twee die aan elkaar tegengesteld zijn. Bij $0$ spreken we af dat de twee tegengestelde wortel samenvallen.
        \begin{proposition}\nl

            \begin{tabular}{@{}l@{ }l}
            Een reëel getal $a$ heeft een vierkantswortel & als en slechts als $a$ positief is (dus $0\leq a$) \\
                                                          & als en slechts als vergelijking $x^2 = a$ een oplossing heeft \\
                                                          & als en slechts als vergelijking $x^2 - a = 0 $ een oplossing heeft 
            \end{tabular}

            Als $a>0$, dan heeft $a$ twee vierkantswortels, een positieve die we noteren met $\sqrt{a}$, en een negatieve die we noteren met $-\sqrt{a}$.
        \end{proposition}

        \begin{quickquestion*}{} Geldt voor elke reëel getal $x$ dat $(\sqrt{x})^2 = x$?\end{quickquestion*}
        \begin{quickquestion*}{} Geldt voor elke reëel getal $x$ dat $(\sqrt{|x|})^2 = x$?\end{quickquestion*}
        \begin{quickquestion*}{} Geldt voor elke reëel getal $x$ dat $(\sqrt{|x|})^2 = |x|$?\end{quickquestion*}
        \begin{quickquestion*}{} Geldt voor elke reëel getal $x$ dat $|\sqrt{|x|}|^2 = |x|$?\end{quickquestion*}
        \begin{quickquestion*}{} Geeft telkens de precieze verzameling $x$'en waarvoor de gelijkheid wel geldt.\end{quickquestion*}


        Er is een evident verband met grafieken van functies: de vergelijking $x^2-a=0$ is oplosbaar als de grafiek van de functie $y=x^2-a$ de $x$-as snijdt, en de nulwaarden of nulpunten zijn precies de twee vierkantswortels van $a$.

    \end{expandable}


    De definitie van vierkantswortels van complexe getallen is volledig gelijkaardig aan de gekende definitie voor reële getallen:
    \begin{definition}[Vierkantswortel van complex getal]\nl

        Een complex getal $w$ is een {vierkantswortel} van een complex getal $z$ als 
        \formulevb{z = w^2}{i \text{ en } -i \text{ zijn wortels van} -1 \text{ want } i^2 = (-i)^2 = -1}
    \end{definition}

    \begin{example}
        \begin{question} Het complex getal $2i$ is \choiceEen vierkantswortel van $-4$. \end{question}
        \begin{question} Het complex getal $-2i$ is \choiceEen vierkantswortel van $-4$. \end{question}
        \begin{question} Het complex getal $2$ is \choiceGeen vierkantswortel van $4$. \end{question}
        \begin{question} Het complex getal $-2$ is \choiceGeen vierkantswortel van $-4$. \end{question}
        \begin{question} Het complex getal $1+2i$ is \choiceEen vierkantswortel van $-1+4i$. \end{question}
        \begin{question} Het complex getal $1-2i$ is \choiceGeen vierkantswortel van $-1+4i$. \end{question}
        \begin{question} Het complex getal $1+i$ is \choiceEen vierkantswortel van $2i$. \end{question}
    \end{example}

    Bij complexe getallen zijn vierkantswortels eenvoudiger dan bij reële getallen, want het verschil tussen positieve en negatieve getallen valt weg.

    \begin{proposition}
        Elke complex getal $w$ heeft twee tegengestelde vierkantswortels.

        of ook, equivalent geformuleerd:
        {\centering

        De vergelijking $x^2-w = 0$ heeft steeds twee aan elkaar tekengestelde wortels.
        }
    \end{proposition}
    We spreken af dat voor $w=0$ het getal $0$ een 'dubbele wortel' is, waarbij $0$ en $-0=0$ tegengestelden zijn\ldots

    \begin{expandable}{proof}{Een erg gedetailleerd bewijs}
        We zullen de eigenschap bewijzen door voor een willekeurig complex getal $w=a+bi$ expliciet twee vierkantswortels te berekenen.

        We zoeken twee oplossingen van de vergelijking $z^2-w=0$. Stel $z=x+iy$, dan zoeken we dus geschikte reële getallen $x$ en $y$ zodat 
        $$
        a+bi = (x+iy)^2 = x^2+2xyi+(yi)^2 = x^2-y^2 +2xyi.
        $$
        Omdat twee complexe getallen enkel aan elkaar gelijk zijn als hun reële delen en imaginaire delen gelijk zijn, volgt dat we oplossingen zoeken van het stelsel
        $$
        (A) \leftrightarrow \begin{cases} x^2 - y^2 = a & (1) \\
                        2xy = b & (2)
        \end{cases}
        $$
        Dit is een stelsel van twee vergelijkingen met twee onbekenden, maar de vergelijkingen zijn niet lineair. De theorie van het oplossen van lineaire stelsels helpt ons dus niet.

        Het oplossen van niet-lineaire stelsels is in het algemeen hopeloos, maar hier is er een trucje.
        Bij gebrek aan betere ideeën, berekenen we $a^2+b^2$ door vergelijkingen (1) en (2) te substitueren
        $$
        a^2+b^2 = (x^2-y^2)^2 + (2xy)^2 = x^4 -2x^2y^2 + y^4 + 4 x^2y^2 = x^4+2x^2y^2 + y^4 = (x^2+y^2)^2
        $$
        Dat is een mooi resultaat, maar helpt het ons ook vooruit?

        We kunnen nu vergelijking $x^2+y^2=\sqrt{a^2+b^2}$ toevoegen aan het stelsel (A) zonder de oplossingen van dat stelsel te wijzigen.

        \begin{quickquestion*}{} Waarom kunnen er geen oplossingen verdwijnen, en ook geen oplossingen bijkomen.\end{quickquestion*}

        We zoeken dus $x$ en $y$ zodat
        \begin{equation}
        \begin{cases} x^2 - y^2 = a & (1) \\
                        2xy = b & (2) \\
                     x^2 + y^2 = \sqrt{a^2+b^2} & (3)
        \end{cases} 
        \end{equation}
        Op het eerste zicht hebben we het probleem moeilijke gemaakt, want nu hebben we zelfs een vergelijking meer om mee rekening te hopuden.
        Maar door de vergelijkingen (1) en (3) bij elkaar op te tellen resp. van elkaar af te trekken, vinden we relatief eenvoudig uitdrukkingen voor $x^2$ en $y^2$:
        $$
        \begin{cases} 2x^2 = a + \sqrt{a^2+b^2} \\
                    -2y^2 = a -\sqrt{a^2+b^2} 
        \end{cases}
        , \text{ of dus } (B) \leftrightarrow 
        \begin{cases} x = \pm\sqrt{\frac{a + \sqrt{a^2+b^2}}{2}} \\
                      y = \pm\sqrt{\frac{-a +\sqrt{a^2+b^2}}{2}} 
        \end{cases}
        $$

        Je kan opmerken dat deze uitdrukkingen zinvol zijn, omdat elke uitdrukking onder een wortelteken steeds zeker positief is.
        Bovendien voldoen de $x$ en $y$ uit de laatste uitdrukkingen bijna automatisch aan de vergelijking $2xy=b$. 
        Dat kunnen we makkelijkst nagaan door $4x^2y^2$ te berekenen (want dan hebben we minder wortels):
        $$
        4x^2y^2 = (a+\sqrt{a^2+b^2}(-a+\sqrt{a^2+b^2})) = -a^2 + (a^2+b^2) = b^2
        $$
        en dus is $2xy=\pm b$. 
        Er zijn in (B) nog twee mogelijkheden voor $x$, en twee voor $y$. Door deze slim te combineren kunne we er voor zorgen dat inderdaad $2xy=b$ (en niet $-b$).
        
        Conclusie: we vinden voor elk complex getal $w=a+bi$ twee vierkantswortels $z=\pm(x+iy)$, namelijk

        Als $b\geq 0$ dan zijn $z_1$ en $-z_1$ wortels met $z_1 = \sqrt{\frac{a + \sqrt{a^2+b^2}}{2}} + i\sqrt{\frac{-a + \sqrt{a^2+b^2}}{2}}$, en 

        als $b< 0$ dan zijn $z_2$ en $-z_2$ wortels met $z_2 = \sqrt{\frac{a + \sqrt{a^2+b^2}}{2}} - i\sqrt{\frac{-a + \sqrt{a^2+b^2}}{2}}$, en 

        Dit beëindigt het bewijs: we hebben voor elk complex getal $w$ expliciet twee wortels berekend.
    \end{expandable}

    \begin{remark}
        Vierkantswortels in $\C$ zijn eigenlijk eenvoudiger dan in $\R$, omdat er \textit{altijd} vierkantswortels bestaan.
        Toch is er een subtiliteit.

        Inderdaad, voor ëen positief reëel getal $a$ hebben we $\sqrt{a}$ gedefinieerd als de \textit{positieve} vierkantswortel, en $-\sqrt{a}$ is dan natuurlijk de negatieve wortel.
        Deze keuze is voor complexe getallen echter onmogelijk, omdat op $\C$ de begrippen 'positief' en 'negatief' geen betekenis hebben.

        Het is dus niet eenvoudig mogelijk om uit de twee wortels van een complex getal $w$ er op consistente wijze één te kiezen om $\sqrt{w}$ te zijn.
        \\
        Naast het probleem met de \hyperref[eig:geen_wortelteken_voor_complexe_getallen]{rekenregels}, is dit een extra reden om \textbf{het gebruik van het symbool $\sqrt{\cdot}$ voor complexe getallen te vermijden}.

    \end{remark}
\end{document}