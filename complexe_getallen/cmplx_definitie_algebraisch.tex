\documentclass{ximera}
\handouttrue
\input{../preamble}
\addPrintStyle{..}
\begin{document}
	\author{Wim Obbels}
	\xmtitle{Complexe getallen (algebraisch)}{}
	% \xmtitle{Complexe getallen (algebraïsch)}{}
	\label{xim:cmplx_definitie_algebraisch}
    
%Met een nieuwe symbool $i$  kunnen we de complexe getallen definiëren.
% Daardoor ontstaan ook combinaties als $2i$ en $i+1$, die we \textbf{complexe getallen} noemen, niet omdat ze ingewikkeld zijn, maar omdat ze \textit{samengesteld} zijn uit een gewoon reëel getal en een veelvoud van het nieuwe getal $i$.


\begin{definition}[Complex getal]\label{def:complex_getal}\label{def:reeel_deel}\label{def:imaginair_deel}\nl
	
    Een \textbf{complex getal} is een uitdrukking van de vorm \(a+bi\), met \(a,b \in \R\).
    \\
    Het symbool \(i\) waarvoor \( \important{i^2=-1}\) noemen we de \textbf{imaginaire eenheid}.
    
De \textbf{verzameling complexe getallen} noteren we met \(\important{\C=\{a+bi \;|\; a,b\in\R;\; i^2=-1\}} \).

Voor een complex getal $z=a+bi$ noemen we: \\
\begin{tabular}{@{\quad}l@{ }ll}    
    & $a=\Re(a+bi)=\Re(z)$ het \textbf{reëel deel}  \\
    & $b=\Im(a+bi)=\Im(z)$ het \textbf{imaginair deel}.
\end{tabular}

\begin{tabular}{@{}l@{ }ll}    
    Als $\Im(z) = b=0$, dan is $z=a$ & een \textbf{(zuiver) re\"eel getal}  \\%& en $\R$ is dus een deelverzameling van $\C$,\\
    Als $\Re(z) = a=0$, dan is $z=bi$ & een \textbf{(zuiver) imaginair getal}.
\end{tabular}
    
    
Twee complexe getallen zijn \textbf{gelijk} indien ze  dezelfde reële en imaginaire delen hebben:
\vspace{-0.35cm}
\begin{center}
    \begin{tabular}{rcl}
        \(z = w\) &\(\iff\)& \(\Re(z)=\Re(w) \Ten \Im(z)=\Im(w)\) \\    
        \(a+bi = c + di \)  &\(\iff\)& \(a = c \Ten b = d \)\\   
    \end{tabular}
\end{center}

\end{definition}

%\begin{expandable}{remark}{Notatie}
    Complexe getallen worden dikwijls met de letters \(z\) of \(w\) genoteerd, net zoals we voor een natuurlijk getal \(n\) gebruiken of voor functies \(f, g \) en \(h\).
%\end{expandable}

\begin{expandable}{remark}{Gelijkheid van complexe getallen}
    Het kan overbodig lijken om te vermelden wanneer twee complexe getallen \(z = a+bi\) en \(w = c+di\) aan elkaar gelijk zijn. Merk op dat een gelijkaardige regel voor breuken \textit{niet} geldt: als $\frac{a}{b} = \frac{c}{d}$ gelijk zijn hoeft niet te gelden dat \(a = c\) en \(b = d\). De breuken $\frac{1}{2}$ en $\frac{3}{6}$ zijn immers gelijk, maar hun tellers en noemers toch zijn verschillend. 
\end{expandable}

% \begin{example}{Enkele complexe getallen met hun imaginair en reëel deel.}
%     \[
%     \begin{array}{c@{\quad}|@{\quad}r@{\quad}r}
%         z & \Re(z) & \Im(z) \\\hline
%         2+3i & 2 & \answer{3} 
%         \\
%         5-6i &\answer{5} & \answer{-6} 
%     \end{array}
%     \qquad
%     \begin{array}{c@{\quad}|@{\quad}r@{\quad}r}
%         z & \Re(z) & \Im(z) \\\hline
%         42 & 42 & 0 
%         \\
%         42i & 0 & 42
%     \end{array}
%     \qquad
%     \begin{array}{c@{\quad}|@{\quad}r@{\quad}r}
%         z & \Re(z) & \Im(z) \\\hline
%         1+i\sqrt{2} & 1 & \sqrt{2}
%         \\
%         1+\pi & 1+\pi & 0 
%         \\
%     \end{array}
%     \]
% De getallen $42$ en $1+\pi$ zijn zuiver reële getallen. Het getal $42i$ is een zuiver imaginair getal. 
% % \\
% \end{example}

\begin{example} Volgende uitdrukkingen zijn allemaal complexe getallen:
    $$
    \def\arraystretch{1.5}
    \begin{array}{c|rr c lll}
        z & \Re(z) & \Im(z) \\\hline
        2+3i & 2 & \answer{3} 
        \\
        5-6i &\answer{5} & \answer{-6} 
        \\
        1+i\sqrt{2} & 1 & \sqrt{2}
        \\
        \dfrac{1-i\sqrt{5}}{2}  & \dfrac{1}{2} & -\dfrac{\sqrt{5}}{2}
        \\
    \end{array}
    \qquad\qquad
    \begin{array}{c|rr c lll}
        z & \Re(z) & \Im(z) \\\hline
        1+\pi & 1+\pi & 0
        \\
        42 & \answer{42} & 0
        \\
        42i & \answer{0} & \answer{42} 
        \\
        5i + 7 & \answer{7} & \answer{5} 
        \\
        2+ 5i + \pi & \answer{2+\pi} & \answer{5} 
        \\
    \end{array}
    $$
    De getallen $42$ en $1+\pi$ zijn zuiver reële getallen, $42i$ is een zuiver imaginair getal. 
    \\
    \end{example}


    \begin{exercise} Geef telkens het reëel en het imaginair deel van de volgende complexe getallen.
        %\begin{xmmulticols}
        \renewcommand{\xmFixFormatLength}{10ex}
        \renewcommand{\xmFixFormatPosition}{c}
        \begin{question}
            \xmFixFormat{\(8+3i\)} heeft reëel deel $\answer{8}$ en imaginair deel $\answer{3}$. 
            \begin{oplossing} 
                Het reëel en imaginair deel van een complex getal vind je door het getal te schrijven in de vorm $a+bi$,
                en dan is 
                \(\Re(a+bi) = a\) en \(\Im(a+bi) = b\).
    
                Hier wordt dat dus \(\Re(8+3i) = 8\) en \(\Im(8+3i) = 3\).
            \end{oplossing}
        \end{question}
        
        \begin{question}
            \xmFixFormat{\(\sqrt{2} - \sqrt{2}i\)} heeft reëel deel $\answer{\sqrt{2}}$ en imaginair deel $\answer{-\sqrt{2}}$. 
        \end{question}
        \begin{question}
            \xmFixFormat{\(7i\)} heeft reëel deel $\answer{0}$ en imaginair deel $\answer{7}$. 
        \end{question}
        \begin{question}
            \xmFixFormat{\(5\)} heeft reëel deel $\answer{5}$ en imaginair deel $\answer{0}$. 
        \end{question}
        \begin{question}
            \xmFixFormat{\(5i-4\)} heeft reëel deel $\answer{-4}$ en imaginair deel $\answer{5}$. 
        \end{question}
        \begin{question}
            \xmFixFormat{\(\frac{1+2i}{3}\)} heeft reëel deel $\answer{\frac{1}{3}}$ en imaginair deel $\answer{\frac{2}{3}}$. 
        \end{question}
        \begin{question}
            \xmFixFormat{\(i\)} heeft reëel deel $\answer{0}$ en imaginair deel $\answer{1}$. 
        \end{question}
        \begin{question}
            \xmFixFormat{\(\frac{9+3i}{3}\)} heeft reëel deel $\answer{3}$ en imaginair deel $\answer{1}$. 
        \end{question}
        \begin{question}
            \xmFixFormat{\(\pi + i + e\)} heeft reëel deel $\answer{\pi+e}$ en imaginair deel $\answer{1}$.
        \end{question}
        \begin{question}
            \xmFixFormat{\(0\)} heeft reëel deel $\answer{0}$ en imaginair deel $\answer{0}$.
        \end{question}
        \begin{question}
            \xmFixFormat{\(\frac{\sqrt{5} + \sqrt{5}i }{\sqrt{5}}\)} heeft reëel deel $\answer{1}$ en imaginair deel $\answer{1}$. 
        \end{question}
        
        %\end{xmmulticols}
        
    \end{exercise}
        
    \begin{exercise} Schrijf zo eenvoudig mogelijk, door gebruik te maken van de gelijkheid $i^2=-1$.
        \renewcommand{\xmFixFormatLength}{3ex}
        \renewcommand{\xmFixFormatPosition}{r}
    
        \begin{xmmulticols}[5]
        \begin{question} \xmFixFormat{\( i^2\)}     = $\answer{-1}$	\end{question}
        \begin{question} \xmFixFormat{\( i^5\)}     = $\answer{i}$	\end{question}
        \begin{question} \xmFixFormat{\(-i^3\)}     = $\answer{i}$	\end{question}
        \begin{question} \xmFixFormat{\( i^4\)}     = $\answer{-1}$	\end{question}
        \begin{question} \xmFixFormat{\(-i^4\)}     = $\answer{-1}$	\end{question}
        \begin{question} \xmFixFormat{\(i^{10}\)}   = $\answer{-1}$	\end{question}
        \begin{question} \xmFixFormat{\(i^{11}\)}   = $\answer{-i}$	\end{question}
        \begin{question} \xmFixFormat{\(i^{12}\)}   = $\answer{1}$	\end{question}
        \begin{question} \xmFixFormat{\(-i^2\)}     = $\answer{1}$	\end{question}
        \begin{question} \xmFixFormat{\(i^{2028}\)} = $\answer{1}$	\end{question}
        \end{xmmulticols}
    \end{exercise}
    

\end{document}

