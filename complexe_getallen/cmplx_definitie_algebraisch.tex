\documentclass{ximera}
\handouttrue
\input{../preamble}
\addPrintStyle{..}
\begin{document}
	\author{Wim Obbels}
	\xmtitle{Complexe getallen (algebraïsch)}{}
	\label{xim:cmplx_definitie_algebraisch}
    
%Met een nieuwe symbool $i$  kunnen we de complexe getallen definiëren.
% Daardoor ontstaan ook combinaties als $2i$ en $i+1$, die we \textbf{complexe getallen} noemen, niet omdat ze ingewikkeld zijn, maar omdat ze \textit{samengesteld} zijn uit een gewoon reëel getal en een veelvoud van het nieuwe getal $i$.


\begin{definition}[Complex getal]\label{def:complex_getal}\label{def:reeel_deel}\label{def:imaginair_deel}\nl
	
    Een \textbf{complex getal} is een uitdrukking van de vorm \(a+bi\), met \(a,b \in \R\).
    \\
    Het symbool \(i\) waarvoor \( \important{i^2=-1}\) noemen we de \textbf{imaginaire eenheid}.
    
De \textbf{verzameling complexe getallen} noteren we met \(\important{\C=\{a+bi \;|\; a,b\in\R;\; i^2=-1\}} \).

Voor een complex getal $z=a+bi$ noemen we: \\
\begin{tabular}{@{\quad}l@{ }ll}    
    & $a=\Re(a+bi)=\Re(z)$ het \textbf{reëel deel}  \\
    & $b=\Im(a+bi)=\Im(z)$ het \textbf{imaginair deel}.
\end{tabular}

\begin{tabular}{@{}l@{ }ll}    
    Als $\Im(z) = b=0$, dan is $z=a$ & een \textbf{(zuiver) re\"eel getal}  \\%& en $\R$ is dus een deelverzameling van $\C$,\\
    Als $\Re(z) = a=0$, dan is $z=bi$ & een \textbf{(zuiver) imaginair getal}.
\end{tabular}
    
    
Twee complexe getallen zijn \textbf{gelijk} indien ze  dezelfde reële en imaginaire delen hebben:
\vspace{-0.35cm}
\begin{center}
    \begin{tabular}{rcl}
        \(z = w\) &\(\iff\)& \(\Re(z)=\Re(w) \Ten \Im(z)=\Im(w)\) \\    
        \(a+bi = c + di \)  &\(\iff\)& \(a = c \Ten b = d \)\\   
    \end{tabular}
\end{center}

\end{definition}

%\begin{expandable}{remark}{Notatie}
    Complexe getallen worden dikwijls met de letters \(z\) of \(w\) genoteerd, net zoals we voor een natuurlijk getal \(n\) gebruiken of voor functies \(f, g \) en \(h\).
%\end{expandable}

\begin{expandable}{remark}{Gelijkheid van complexe getallen}
    Het kan overbodig lijken om te vermelden wanneer twee complexe getallen \(z = a+bi\) en \(w = c+di\) aan elkaar gelijk zijn. Merk op dat een gelijkaardige regel voor breuken \textit{niet} geldt: als $\frac{a}{b} = \frac{c}{d}$ gelijk zijn hoeft niet te gelden dat \(a = c\) en \(b = d\). De breuken $\frac{1}{2}$ en $\frac{3}{6}$ zijn immers gelijk, maar hun tellers en noemers toch zijn verschillend. 
\end{expandable}

% \begin{example}{Enkele complexe getallen met hun imaginair en reëel deel.}
%     \[
%     \begin{array}{c@{\quad}|@{\quad}r@{\quad}r}
%         z & \Re(z) & \Im(z) \\\hline
%         2+3i & 2 & \answer{3} 
%         \\
%         5-6i &\answer{5} & \answer{-6} 
%     \end{array}
%     \qquad
%     \begin{array}{c@{\quad}|@{\quad}r@{\quad}r}
%         z & \Re(z) & \Im(z) \\\hline
%         42 & 42 & 0 
%         \\
%         42i & 0 & 42
%     \end{array}
%     \qquad
%     \begin{array}{c@{\quad}|@{\quad}r@{\quad}r}
%         z & \Re(z) & \Im(z) \\\hline
%         1+i\sqrt{2} & 1 & \sqrt{2}
%         \\
%         1+\pi & 1+\pi & 0 
%         \\
%     \end{array}
%     \]
% De getallen $42$ en $1+\pi$ zijn zuiver reële getallen. Het getal $42i$ is een zuiver imaginair getal. 
% % \\
% \end{example}

\begin{example} Volgende uitdrukkingen zijn allemaal complexe getallen:
    $$
    \def\arraystretch{1.5}
    \begin{array}{c|rr c lll}
        z & \Re(z) & \Im(z) \\\hline
        2+3i & 2 & \answer{3} 
        \\
        5-6i &\answer{5} & \answer{-6} 
        \\
        1+i\sqrt{2} & 1 & \sqrt{2}
        \\
        \dfrac{1-i\sqrt{5}}{2}  & \dfrac{1}{2} & -\dfrac{\sqrt{5}}{2}
        \\
    \end{array}
    \qquad\qquad
    \begin{array}{c|rr c lll}
        z & \Re(z) & \Im(z) \\\hline
        1+\pi & 1+\pi & 0
        \\
        42 & \answer{42} & 0
        \\
        42i & \answer{0} & \answer{42} 
        \\
        5i + 7 & \answer{7} & \answer{5} 
        \\
        2+ 5i + \pi & \answer{2+\pi} & \answer{5} 
        \\
    \end{array}
    $$
    De getallen $42$ en $1+\pi$ zijn zuiver reële getallen, $42i$ is een zuiver imaginair getal. 
    \\
    \end{example}

\begin{exercise}{Bepaal het imaginair en reëel deel van de volgende complexe getallen}
\begin{xmmulticols}[4]
\begin{question}
    \(8+3i\)
    \begin{oplossing} \(\Re(8+3i) = 8\) en \(\Im(8+3i) = 3\)\end{oplossing}

\end{question}

\begin{question}
    \(\sqrt{2} + \sqrt{2}i\)
\end{question}
\begin{question}
    \(7i\)
\end{question}
\begin{question}
    \(\frac{1+2i}{3}\)
\end{question}
\begin{question}
    \(1\)
\end{question}
\begin{question}
\(i\)
\end{question}
\begin{question}
    \(\frac{9+3i}{3}\)
\end{question}
\begin{question}
    \(\pi + e + i \)
\end{question}
\begin{question}
    \(0\)
\end{question}
\begin{question}
    \(\frac{\sqrt{5} + \sqrt{5}i }{\sqrt{5}}\)
\end{question}

\end{xmmulticols}

\end{exercise}

\begin{exercise}{Vereenvoudig.}
\begin{xmmulticols}[6]
\begin{question}
    \(i^2\)
\end{question}
\begin{question}
    \(i^5\)
\end{question}
\begin{question}
    \(-i^3\)
\end{question}
\begin{question}
    \(i^4\)
\end{question}
\begin{question}
    \(-i^4\)
\end{question}
\begin{question}
    \(i^{10}\)
\end{question}
\begin{question}
    \(i^{11}\)
\end{question}
\begin{question}
    \(-i^{12}\)
\end{question}
\begin{question}
    \(-i^2\)
\end{question}
\begin{question}
    \(-i^3\)
\end{question}
\end{xmmulticols}
\end{exercise}

% \begin{exercise}{Ga na of devolgende complexe getallen aan elkaar gelijk zijn.}
% \begin{xmmulticols}

% \begin{question}
%     \(\) en \(\)
% \end{question}
% \begin{question}
%     \(\) en \(\)
% \end{question}
% \begin{question}
%     \(\) en \(\)
% \end{question}
% \begin{question}
%     \(\) en \(\)
% \end{question}
% \begin{question}
%     \(\) en \(\)
% \end{question}
% \begin{question}
%     \(\) en \(\)
% \end{question}
% \begin{question}
%     \(\) en \(\)
% \end{question}
% \begin{question}
%     \(\) en \(\)
% \end{question}
% \begin{question}
%     \(\) en \(\)
% \end{question}
% \begin{question}
%     \(\) en \(\)
% \end{question}
% \begin{question}
%     \(\) en \(\)
% \end{question}
% \end{xmmulticols}
% \end{exercise}


% \begin{example} De volgende uitdrukkingen zijn complexe getallen:
%     $$
% \def\arraystretch{1.5}
% \begin{array}{c|cc@{\qquad} @{\qquad}c|@{\qquad}cc@{\qquad} @{\qquad}c@{\qquad}cc}
%      & \Re(z) & \Im(z) 
%     &  & \Re(z) & \Im(z) 
%     &  & \Re(z) & \Im(z) 
%     \\[2mm]\hline
%      2+3i & 2 & \answer{3} 
%     &17 & \answer{17} & 0
%     &i\sqrt{5} + 3  & 3 & {\sqrt{5}}
%     \\
%     1+\sqrt{2}i & 1 & \sqrt{2}
%     &17i & \answer{0} & \answer{17} 
%     &1+\pi & 1+\pi & 0
%     \\
% \end{array}
% $$
% De getallen $42$ en $1+\pi$ zijn zuiver reële getallen.\\
% Het getal $42i$ is een zuiver imaginair getal. 
% % \\
% Van het complex getal $2+3i$ is het reëel deel $2$ en het imaginair deel $3$. 
%\end{example}

    % \begin{example} De volgende uitdrukkingen zijn complexe getallen:
        %     $$
% \def\arraystretch{1.5}
% \begin{array}{c@{\quad}|@{\quad}r@{\quad}r}
%     z & \Re(z) & \Im(z) \\\hline
%     2+3i & 2 & \answer{3} 
%     \\
%     1+\sqrt{2}i & 1 & \sqrt{2}
%     \\
%     17 & \answer{17} & 0
%     \\
%     \dfrac{1-i\sqrt{5}}{2}  & \dfrac{1}{2} & -\dfrac{\sqrt{5}}{2}
%     \\
%     17i & \answer{0} & \answer{17} 
%     \\
%     1+\pi & 1+\pi & 0
%     \\
% \end{array}
% $$
% De getallen $42$ en $1+\pi$ zijn zuiver reële getallen.\\
% Het getal $42i$ is een zuiver imaginair getal. 
% % \\
% % Van het complex getal $2+3i$ is het reëel deel $2$ en het imaginair deel $3$. 
% \end{example}


% VERSIE PAPA 


% \begin{example} De volgende uitdrukkingen zijn complexe getallen:
%     $$
% \def\arraystretch{1.5}
% \begin{array}{c@{\quad}|@{\quad}r@{\quad}r}
%     z & \Re(z) & \Im(z) \\\hline
%     2+3i & 2 & \answer{3} 
%     \\
%     5-6i &\answer{5} & \answer{-6} 
%     \\
%     1+i\sqrt{2} & 1 & \sqrt{2}
%     \\
%     \dfrac{1-i\sqrt{5}}{2}  & \dfrac{1}{2} & -\dfrac{\sqrt{5}}{2}
%     \\
% \end{array}
% \qquad\qquad
% \begin{array}{c@{\quad}|@{\quad}r@{\quad}r}
%     z & \Re(z) & \Im(z) \\\hline
%     1+\pi & 1+\pi & 0
%     \\
%     42 & \answer{42} & 0
%     \\
%     42i & \answer{0} & \answer{42} 
%     \\
%     5i + 7 & \answer{7} & \answer{5} 
%     \\
%     2+ 5i + \pi & \answer{2+\pi} & \answer{5} 
%     \\
% \end{array}
% $$
% De getallen $42$ en $1+\pi$ zijn zuiver reële getallen, en $42i$ is een zuiver imaginair getal. 
% \\
% Van het complex getal $2+3i$ is het reëel deel $2$ en het imaginair deel $3$. 
% \end{example}

% Alternatief (test)
% \begin{example} Volgende uitdrukkingen zijn allemaal complexe getallen:
% $$
% \begin{array}{lll c lll}
% 2+3i & \Re(2+3i) = 2 & \Im(2+3i) = \answer{3} 
% \\
% 42 & \Re(42) = \answer{42} & \Im(42) = 0
% \\
% 42i & \Re(42i) = \answer{0} & \Im(42i) = \answer{42} 
% \\
% 1+\pi & \Re(1+\pi) = 1+\pi & \Im(1+\pi) =  0
% \\
% 1+i\sqrt{2} & \Re(1+i\sqrt{2}) = 1 & \Im(1+i\sqrt{2}) = \sqrt{2}
% \\
% \dfrac{1-i\sqrt{5}}{2}  & \Re(\dfrac{1-i\sqrt{5}}{2}) = \dfrac{1}{2} & \Im(\dfrac{1-i\sqrt{5}}{2}) = -\dfrac{\sqrt{5}}{2}
% \\
% \end{array}
% $$

% De getallen $42$ en $1+\pi$ zijn zuiver reële getallen, en $42i$ is een zuiver imaginair getal. 
% \\
% Van het complex getal $2+3i$ is het reëel deel $2$ en het imaginair deel $3$. 
% \end{example}

% Alternatief (test)
% \begin{example} Volgende uitdrukkingen zijn allemaal complexe getallen:
% \begin{multicols}{3}
% \begin{enumerate}
% \item $2+3i$ 				    % a
% \item $42$   				    % b
% \item $42i$  				    % c
% \item $1+\pi$				    % d
% \item $1+i\sqrt{2}$ 		    % e
% \item $\dfrac{1-i\sqrt{5}}{2}$  % f
% \end{enumerate}
% \end{multicols}
% De getallen $42$ en $1+\pi$ zijn zuiver reële getallen, en $42i$ is een zuiver imaginair getal. 
% \\
% Van het complex getal $2+3i$ is het reëel deel $2$ en het imaginair deel $3$. 
% \end{example}


\end{document}

