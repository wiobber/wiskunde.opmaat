\documentclass{ximera}
\input{../preamble}
\addPrintStyle{..}
\begin{document}
	\author{Wim Obbels}
	\xmtitle{Wiskunde Op Maat}{}
	\label{xim:xuitleg_leerkrachten}

    Dit handboek over complexe getallen is het eerste deel van een reeks die hopelijk ooit de volledige leerstof van minstens de derde graad van het Vlaamse secundair onderwijs zal omvatten.

    De belangrijkste kenmerken zijn
    \begin{enumerate}
        \item erg kwaliteitsvol leermateriaal wiskunde, met duidelijke en exacte formuleringen, verhelderende en motiverende voorbeelden en zowel voldoende eenvoudige inoefen-oefeningen als uitdagende toepassingen.
        \item een minimale impact op de lessen: er wordt materiaal aangeboden dat kan dienen als referentie, studiemateriaal, oefeningen of werkbladen, maar de leerkracht kiest hoe hij wat al dan niet gebruikt. In het bijzonder is er een klassiek handboek, dat zowel in PDF als uitgeprint kan worden aangeboden, eventueel in kleine onderdelen naarmate de lessen vorderen. Bovendien is er een website met hetzelfde materiaal, eventueel uitgebreid met extra voorbeelden, oefeningen of kennisclips.
        \item geschreven in \LaTeX, de lingua franca waarin wiskundige teksten worden geschreven. Het systeem is vrij beschikbaar, is erg krachtig, en wordt over heel de wereld door ontelbare wiskundigen en leerkrachten wiskunde gebruikt. Er is een enorme online community die de technology veder ontwikkeld en al het materiaal vrij ter beschikking stelt.
        \item van dezelfde \LaTeX-broncode worden zowel hoogwaardige PDF-bestanden gemaakt, als interactie webpagina's. Hiervoor wordt gebruik gemaakt van Ximera, oorspronkelijk ontwikkeld aan Ohio State University, en sinds 2019 aan de KU Leuven verder uitgebreid. Ook hiervan is de broncode publiek beschikbaar.
        \item ook de broncode van het hele handboek, inclusief de oefeningen, in vrij beschikbaar. Dit laat elke geïnteresseerde leerkracht met voldoende \LaTeX\ kennis toe om het handboek volledig naar eigen smaak en voorkeuren aan te passen, in te perken of uit te breiden.
        \item het volledige leerplan zal enkel kunnen gerealiseerd worden als er een actieve community ontstaat van gemotiveerde leerkrachten die materiaal ontwikkelen en ter beschikking willen stellen van hun collega's. Zoals gebruikelijk in succesvolle open source projecten, zorgt de structuur van deze community voor een automatische kwaliteitscontrole. Er zal geen wildgroei ontstaan van ongerelateerde oefeningen en werkbladen met diverse notaties en onduidelijke verwachte voorkennis. 
        \item de broncode is erg modulair opgebouwd, zodat van hetzelfde handboek automatisch verschillende versies worden gegenereerd, zogenaamde 'flavours', met meer of minder basisoefeningen, al dan niet historische noten of uitdagende denkvragen, meer of minder bewijzen en meer of minder uitgebreide toepassingen. Om hiervan gebruik te maken is geen enkele technische kennis vereist. Hierdoor ontstaat een \textbf{handboek op maat van de leerkracht}, die precies kan aangeven welke bewijzen, voorbeelden en oefeningen moeten worden opgenomen.
        \item zowel de PDF als de website kunnen hyperlinks bevatten naar meer voorbeelden, meer oefeningen, een alternatieve uitleg, een extra kennisklip of geogebra applet. Hierdoor ontstaat een \textbf{handboek op maat van elke leerling}, die voor de moeilijke onderdelen steeds extra materiaal vindt.
        \item het gebruik van \LaTeX\ garandeert een grote duurzaamheid, minstens wat het genereren van kwaliteitsvolle PDF's betreft: het is hoogst waarschijnlijk dan binnen 10, 20 en 30 jaar dezelfde broncode nog steeds dezelfde PDF zal genereren (net zoals vandaag de meeste \LaTeX bestanden van de jaren 1980 nog perfect bruikbaar zijn, hoewel ze natuurlijk geen gebruik maken van de ontelbare nieuwigheden die sinds die tijd zijn toegevoegd). De webtechnologie zal natuurlijk sneller evolueren, maar het is erg aannemelijk dat in toekomst van de huidige \LaTeX-code automatisch of semi-automatisch nieuwe interactieve applicaties zullen kunnen gemaakt worden.
    \end{enumerate}

    Enkele didactisch-technische kenmerken zijn
    \begin{enumerate}
        \item een duidelijke structuur, met kleurcodes (die eventueel per leerkracht/school kunnen worden aangepast). Zo zijn definities en eigenschappen steeds bondig en correct geformuleerd in groene kaders, en hebben voorbeelden en oefeningen een blauwe achtergrond. Inleidingen, motiveringen, achtergrondinformatie, uitweidingen kunnen naar wens worden beperkt, weggelaten of uitklapbaar gemaakt. Hierdoor wordt voor leerlingen automatisch duidelijk wat er precies moet gekend zijn.
        \item (optioneel) bij theorie worden regelmatig zogenaamde 'vlugge vragen' voorzien, die leerlingen toelaten om na te gaan of ze definities en eigenschappen correct hebben begrepen. Deze vragen kunnen bij een eerste kennismaking wat tijd en moeite vragen, maar moeten naarmate men vertrouwd raakt met de leerstof 'evident' worden voor iedereen. Daarnaast zijn er 'denkvragen' die soms erg algemeen zijn, en dikwijls geen eenduidige antwoorden hebben, maar verwondering of creativiteit kunnen uitlokken.
        \item er wordt gestreefd naar wiskundige exactheid zonder nodeloos formalisme. Definities zijn duidelijk en correct. Notaties die worden ingevoerd worden ook verder gebruikt en herhaald, zodat leerlingen er mee vertrouwd raken. Waar er zinvolle verbanden zijn tussen onderdelen, worden die ook gelegd.
        \item bij elk topic hoor zowel een 'steekkaart' met veronderstelde voorkennis (die eventueel kan worden herhaald/geactiveerd), als een 'steekkaart' met alle belangrijke concepten en eigenschappen (die kan worden gebruikt bij het studeren en bij het maken van oefeningen).
    \end{enumerate}

\end{document}

