\documentclass{ximera}
\input{../preamble}
\addPrintStyle{..}
\begin{document}
	\author{Wim Obbels}
	\xmtitle{De deling van complexe getallen}{}
	\label{xim:cmplx_deling}

    Uit de formule $z\overline{z} = |z|^2$ en de redenering
    $$
    z\overline{z} = |z|^2 \iff \frac{z\overline{z}}{|z|^2}=1 \iff z\cdot \frac{\overline{z}}{|z|^2}=1 
    $$
    volgt dat $\frac{\overline{z}}{|z|^2}$ de inverse is van $z$ (want hun product is $1$.)

    \begin{proposition} Voor een complex getal $z$ geldt
    \formulevb{z^{-1} = \dfrac{1}{z} = \dfrac{\overline{z}}{|z|^2}}{(3+4i)^{-1}=\dfrac{1}{3+4i} = \dfrac{3-4i}{25}= \frac{3}{25}-\frac{4}{25}i}
    \end{proposition}

    Hieruit volgt ook onmiddellijk een manier om het quotiënt van twee complexe getallen te berekenen:

    \begin{proposition} Voor een complexe getal $z$ en $w$ geldt
        \formulevb{\ds\frac{w}{z} = \frac{w\cdot\overline{z}}{|z|^2}}{\ds\frac{1+2i}{3+4i} = \frac{(1+2i)(3-4i)}{25}= \frac{11}{25}+\frac{2}{25}i}
    \end{proposition}
    
    %In het bijzonder zijn zowel de som als het product van een getal en zijn complex toegevoegde steeds \textit{reëel}. Dat $z\overline{z}$ een reëel getal is kunnen we gebruiken om het quotiënt van twee complexe getallen te berekenen. We nemen als voorbeeld
\begin{example} Bereken $\ds\frac{3+2i}{1-5i} =\answer{-\frac{7}{26}+\frac{7}{26}i}$.

    \begin{oplossing}
    De deling van deze twee complexe getallen is opnieuw een complex getal. De deling uitvoeren wil zeggen $\ds \frac{3+2i}{1-5i}$ schrijven in de vorm $a+bi$. Als we teller en noemer vermenigvuldigen met het complex toegevoegde van de noemer verdwijnt de $i$ in de noemer en krijgen we een complex getal van de vorm $a+bi$:
    $$
    \frac{3+2i}{1-5i}= \frac{(3+2i)(1+5i)}{(1-5i)(1+5i)}=\frac{3+15i+2i+10 i^2}{1+5i-5i-25 i^2}=\frac{-7+17i}{26}= - \frac{7}{26} + \frac{17}{26}i
    $$
    %Dus: om een complex getal te verdrijven uit de noemer, vermenigvuldig je teller en noemer met het toegevoegde. %een reëel getal is ook een complex...
    %\\
    Merk op dat dit volledig analoog is met het verdrijven van wortelvormen als $2+3\sqrt{2}$ uit de noemer door teller en noemer te vermenigvuldigen met de zogenaamde \textit{toegevoegde tweeterm} $2-3\sqrt{2}$.

    \end{oplossing}
\end{example}

%We hebben voorlopig niet echt gesproken over het delen van complexe getallen, omdat we dat probleem nu eenvoudig kunnen reduceren tot het berkene van de inverse, waarvoor we al een makkelijke formule hebben gevonden. Inderdaad, uit \ref{eig:modulus:invers} hierboven blijkt dat het quotiënt van twee complex getallen kan worden uitgedrukt met behulp van het complex toegevoegde en de modulus:
% 
%\begin{proposition}[Deling en inverse van complexe getallen]\nl 
% De inverse $z^{-1}$ (of $\dfrac 1z$) van een complex getal $z=a+bi$ wordt gegeven door
%  $$
%  z^{-1 } = \dfrac{\overline{z}}{|z|^2}
%  $$
%  of
%  $$
%  z^{-1 }= (a+bi)^{-1} = \frac{1}{a+bi} = \frac{a-bi}{a^2+b^2}
%  $$
%  Omdat delen door een complex getal hetzelfde is als vermenigvuldigen met zijn inverse, vinden we dat delen hetzelfde is als vermenigvuldigen met het complex toegevoegde en dan delen door de modulus in het kwadraat. Voor getallen $z_1=a+bi$ en $z_2=c+di$ geldt dus 
%  $$
%    \frac{z_1}{z_2} = z_1 \cdot z_2^{-1 }= (a+bi)(c+di)^{-1} = \frac{a+bi}{c+di} = \frac{(a+bi)(c+di)}{c^2+d^2}
%  $$
%  Of nog: om een complex $z$ getal uit de noemer van een breuk te verwijderen, volstaat het de breuk te delen en vermenigvuldigen met het complex toegevoegde $\overline{z}$:
%  $$
%  \frac{1}{a+bi} = \frac{1(a-bi)}{(a+bi)(a-bi)} = \frac{a-bi}{a^2+b^2}
%  $$
%\end{proposition}

\begin{exercise} Schrijf volgende uitdrukkingen als een complex getal van de vorm $a+bi$:
%	\begin{question} $(a+bi)\cdot\frac{a-bi}{a^2+b^2} = \answer[onlineshowanswerbutton]{1}$
%	\end{question}
	\begin{question} $\dfrac{1}{1+2i} =  \answer[onlineshowanswerbutton]{\frac15 -\frac25 i}$
	\begin{oplossing}
	Vermenigvuldig teller en noemer met $\overline{1+2i} = 1-2i$, zodat de noemer $(1-2i)(1+2i) = 5\in\R$:
	$$
	\frac{1}{1+2i} = \frac{1}{1+2i} \frac{1-2i}{1-2i} = \frac{1-2i}{5}=\frac15 -\frac25 i
	$$
	\end{oplossing}
	\end{question}
    \begin{question} $\dfrac{1+2i}{3+4i} =  \answer[onlineshowanswerbutton]{\frac{11}{25} + \frac{2}{25}i}$
    \begin{oplossing}
	Vermenigvuldig teller en noemer met $\overline{3+4i} = 3-4i$, zodat de noemer $(3+4i)(3-4i) = 25\in\R$:  
	$$
	\frac{1+2i}{3+4i} = \frac{1+2i}{3+4i} \cdot \frac{3-4i}{3-4i} = \frac{11 + 2i}{25}= \frac{11}{25} + \frac{2}{25}i
	$$
    \end{oplossing}
    \end{question}	
    \begin{question} $\dfrac{3+4i}{1+2i} =  \answer[onlineshowanswerbutton]{\frac{11}{5}-\frac{2}{5}i}$
    \begin{oplossing}
	Vermenigvuldig teller en noemer met $\overline{1+2i} = 1-2i$, zodat de noemer $(1+2i)(1-2i) = 5\in\R$:
	$$
	\frac{3+4i}{1+2i} = \frac{3+4i}{1+2i} \cdot \frac{1-2i}{1-2i} = \frac{11 - 2i}{5}=\frac{11}{5}-\frac{2}{5}i
	$$ 	
    \end{oplossing}
    \end{question}	
\end{exercise}

\end{document}

