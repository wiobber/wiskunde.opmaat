8\documentclass{ximera}
%\handouttrue
\input{../preamble}
\addPrintStyle{..}
\begin{document}
	\author{Wim Obbels}
	\xmtitle{Werkblad: deling van complexe getallen}{}

    Het optellen en vermenigvuldigen van complexe getallen $a+bi$ en $c+di$ stelt geen probleem zodra je ermee vertrouwd bent dat je steeds  $i^2$ moet vervangen door $-1$.
    
    \begin{exercise}[Opwarming]\nl
        \begin{xmmulticols}
        \begin{question} $(2+3i) + (4+5i)     = \answer{6+8i}$  \end{question}
        \begin{question} $(2+3i) \cdot i      = \answer{-3+2i}$ \end{question}
        \begin{question} $i^3 \cdot i^5       = \answer{1}$     \end{question}
        \begin{question} $(2+3i) \cdot (4+5i) = \answer{-7+22i}$\end{question}
        \end{xmmulticols}
    \end{exercise}

    Op dit werkblad onderzoek je zelf hoe je twee complexe getallen door elkaar zou kunnen delen.

    \begin{denkvraag*}{}
        Hoe zou je het quotiënt $\frac{2+3i}{4+5i} = \answer {123}$ kunnen berekenen ?
        \\
        Zijn er misschien eenvoudige complexe getallen waar je wel kan door delen?
        \\
        Kan je het probleem misschien herformuleren ...?
    \end{denkvraag*}


    {
    \pdfOnly{
        \DeclareTcbTheorem{xexercise}{Opdracht}{
            boxrule=0pt,
            boxsep=0pt,
            colback={\colorexample!0!white},
            colframe={red!0!white},
            % colbacktitle=\colorexample!0!white,
            %borderline west={2pt}{0pt}{red!50!black},
            %coltitle=red,
            coltitle=black,
            fonttitle=\bfseries,
            sharp corners,
            before skip=10pt,
            after skip=10pt,
            breakable,
            before upper={\vspace*{-0.45cm}},
    }{opdracht}
    }
    

    \begin{expandable}{exercise}{Eerste mogelijkheid om het probleem aan te pakken}
        Probeer \textit{een eenvoudiger geval}. Hier kan je bijvoorbeeld proberen een complex getal alvast te delen door een \textit{reëel} getal.
        
        \begin{hint}
            $\frac{2+3i}{4} = \answer{\frac{1}{2}+\frac{3}{4}i}$
            \begin{feedback}[correct]
                Inderdaad, delen door een reëel getal is eenvoudig.
            \end{feedback}
        \end{hint}
        \begin{oplossing}
            Je deelt het reëel deel, en je deelt het imaginair deel:
            $\frac{2+3i}{4} = \frac{1}{2}+\frac{3}{4}i$.

            Je kan nu een complex getal wel delen door een reëel getal, maar nog niet door een ander complex getal.
            Je zal een andere strategie moeten toepassen \ldots
        \end{oplossing}
        % \begin{question} Als de teller $1$ is: $\frac{1}{4+5i} = \answer{123}$
        %     \begin{oplossing}
        %         Dit is eenvoudiger, en een oplossing zou volstaan om oefening 1 op te lossen.
                
        %         Je gebruikt dat $\frac{a}{b} = a \cdot b^{-1}$ om het algemeen probleem van een deling te reduceren tot de vermenigvuldiging (die je al kent), en het berekenen van een inverse (dat je alsnog moet trachten op te lossen).
        %     \end{oplossing}
        % \end{question}
    \end{expandable}

    \begin{expandable}{exercise}{Tweede mogelijkheid om het probleem aan te pakken}
        
        Een \textit{naam} geven aan het gezochte antwoord opent soms nieuwe mogelijkheden.

        \begin{hint} Je zoekt een complex getal $z=x+iy$ zodat 
            $
            \frac{2+3i}{4+5i} = x+iy
            $.

            Kan je dit omvormen tot een stelsel van twee vergelijkingen in twee onbekenden?
        \end{hint}
        \begin{hint} Door de noemer naar het rechterlid te brengen, krijg je
            $2+3i = (x+iy)(4+5i)$, of dus $2+3i = 4x-5y + (5x+4y)i$.

            Zie je hier een stelsel in dat je kan oplossen ?
        \end{hint}
        \begin{hint} Twee complexe getallen zijn aan elkaar gelijk als ze gelijke reële delen hebben, en gelijke imaginaire delen.
            Hier wordt dat dus
                $$
                \begin{cases}
                    4x - 5y = 2   \\
                    5x + 4y = 3
                \end{cases}
                $$
                Los dit stelsel op om het quotiënt te berekenen.
            \end{hint}
        \begin{oplossing}
                \begin{align*}
                   \frac{2+3i}{4+5i} = x+iy & \iff 2+3i = (4+5i)(x+iy) \\
                                            & \iff 2+3i = 4x+4yi + 5xi + 5yi^2  \\
                                            & \iff 2+3i = 4x+4iy + 5xi - 5y  \\
                                            & \iff 2+3i = (4x - 5y) + (4y + 5x)i   \\
                                            & \iff \begin{cases}
                                                 4x - 5y = 2   \\
                                                 5x + 4y = 3
                                                    \end{cases} \\
                \end{align*}
                De oplossing van het stelsel is $x=\frac{23}{41}$ en $y=\frac{2}{41}$, en dus is 
                $
                \frac{2+3i}{4+5i} = \frac{23}{41} + \frac{2}{41}i
                $.

                Hiermee heb je dus een manier om een quotiënt te berekenen: je kan het probleem oplossen met een stelsel.
                In de volgende opgave zal je echter merken dat er een eenvoudigere methode is.
            \end{oplossing}
    \end{expandable}

    \begin{expandable}{exercise}{Derde mogelijkheid om het probleem aan te pakken}

        Een \textit{gelijkaardig} probleem dat je al kan oplossen geeft misschien inspiratie.

        \begin{hint} 
        Herinner je de manier om een wortel uit de noemer te verdrijven, bijvoorbeeld in
        $
        \frac{1+\sqrt{3}}{1-\sqrt{2}} = \answer{123}
        $.

        Kan dit helpen om te delen door een complex getal?
        \end{hint}
        \begin{hint}
            Je kan wortels verdrijven door te vermenigvuldigen met de \textit{toegevoegde tweeterm} van de noemer
            $$
            \frac{1+\sqrt{3}}{1-\sqrt{2}} = \frac{(1+\sqrt{3})(1+\sqrt{2})}{(1-\sqrt{2})(1+\sqrt{2})}
            $$
            omdat de noemer dan erg vereenvoudigt.
        \end{hint}
    \end{expandable}

    \begin{expandable}{proposition}{Oplossing: hoe delen door een complex getal?}

        Door teller en noemer te vermenigvuldigen met het 'toegevoegde' van de noemer, wordt die noemer reëel, en kan je de deling dus makkelijk uitvoeren:

        %Bereken $\frac{2+3i}{4+5i}$.
        %    \begin{oplossing}
                \begin{align*}
                   \frac{2+3i}{4+5i} & = \frac{2+3i}{4+5i} \cdot \frac{4-5i}{4-5i} \\
                                     & = \frac{(2+3i)(4-5i)}{(4+5i)(4-5i)}  \\
                                     & = \frac{8-15+10i+ 12i}{16+25}  \\
                                     & = \frac{-7+22i}{41} =\blue{\frac{-7}{41}+\frac{22}{41}i} 
                \end{align*}
         %   \end{oplossing}
    \end{expandable}

    }  % end redefine exercise

    Met bovenstaande eigenschap kan je dus relatief eenvoudig quotiënten van complexe getallen berekenen.

    \begin{exercise}
        Bereken.
        % Bron: Ingmar    
        \begin{xmmulticols}
        \begin{question} $\dfrac{1}{1-i}$  $= \answer{\frac{1+i}{2}}$
        \end{question}
        \begin{question} $\dfrac{i}{1+i}$ $= \answer{\frac{1+i}{2}}$
        \end{question}
        \begin{question} $\dfrac{1-i}{2i}$ $= \answer{\frac{-1-i}{2}}$
        \end{question}
        \begin{question} $\dfrac{7-6i}{i}$ $= \answer{-6-7i}$
        \end{question}
        \begin{question} $\dfrac{8+i}{i+4}$ $= \answer{\frac{33-4i}{17}}$
        \end{question}
        \begin{question} $\dfrac{-1+5i}{5+i}$ $= \answer{i}$
        \end{question}
    \end{xmmulticols}
    \end{exercise}
        
\end{document}

