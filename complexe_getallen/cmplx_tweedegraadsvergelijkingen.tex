\documentclass{ximera}
\input{../preamble}
\addPrintStyle{..}

\begin{document}
	\author{Wim Obbels}
	\xmtitle{Tweedegraadsvergelijkingen in de complexe getallen}{}
	\label{xim:cmplx_tweedegraadsvergelijkingen}

    Wiskundigen zijn dikwijls erg enthousiast over complexe getallen omdat ze de wiskunde vereenvoudigen.
    \\
    Dit lijkt voor sommigen misschien moeilijk te geloven, maar we zullen het proberen aan te tonen door het oplossen van tweedegraadsvergelijkingen opnieuw te bestuderen nu we stilaan vertrouwd raken met de complexe getallen.

    Voor wie het nodig heeft, herhalen we hier kort wat je enkele jaren geleden hierover hebt geleerd.

    \begin{expandable}{remark}{Opfrissing: vierkantsvergelijkingen $ax^2+bx+c=0$ oplossen in $\R$.}\nl

        \begin{proposition}\nl

            Een (reële) tweedegraadsvergelijking van de vorm $ax^2+bx+c$, met $a,b,c\in\R$ en $a\neq0$
            heeft als \textbf{discriminant} het (reële) getal  \important{D=b^2-4ac}, 
            en heeft als oplossingen

            \begin{tabular}{lll}
                als $D<0$: & geen reële oplossingen \\
                als $D=0$: & precies een reële oplossing, namelijk \important{x_1=-\dfrac{b}{2a}} \\
                als $D>0$: & precies twee reële oplossingen, namelijk 
                             \important{x_1=\dfrac{-b+\sqrt{D}}{2a}} en \important{x_2=\dfrac{-b-\sqrt{D}}{2a}}
            \end{tabular}

            Bovendien zijn de volgende uitspraken equivalent:

            \begin{tabular}{lll}
                (a) & $x_1$ en $x_2$ zijn oplossingen van de vergelijking $ax^2+bx+c=0$ \\
                (b) & $x_1$ en $x_2$ zijn nulpunten  van de functie $f(x)=ax^2+bx+c$ \\
                (c) & $x_1$ en $x_2$ zijn snijpunten van de kromme $y=ax^2+bx+c$ met de $x$-as \\
                (d) & $ax^2+bx+c = a(x-x_1)(x-x_2)$ & (ontbinden in factoren)\\
                (e) & $x_1+x_2 = -\dfrac{b}{a}$ en $x_1\cdot x_2 = \frac{c}{a}$ & (som en product van de wortels)
            \end{tabular}

        \end{proposition}

    \end{expandable}

    Een negatieve discriminant stelt bij de reële getallen een onoverkomelijk probleem, want uit negatieve getallen kan je geen vierkantswortels trekken.
    Bij complexe getallen verdwijnt dat probleem:

    \begin{proposition}\nl

        Een (reële) tweedegraadsvergelijking van de vorm $ax^2+bx+c$, met $a,b,c\in\R$ en $a\neq0$
        heeft altijd twee complex toegevoegde oplossingen, namelijk 
        \begin{center}
            \important{x_1=\dfrac{-b+\sqrt{b^2-4ac}}{2a}} en \important{x_2=\dfrac{-b-\sqrt{b^2-4ac}}{2a}}
        \end{center}
        die echter samenvallen als $b^2-4ac=0$, en dan gelijk worden aan $x_1=x_2=\dfrac{-b}{2a}$.
    \end{proposition}
    Het begrip discriminant wordt dus in zekere zin overbodig (of minstens veel minder belangrijk).


    \begin{example} 
        \begin{question} De wortels van de vergelijking $x^2+4=0$ zijn $x_1=2i$ en $x_2=-2i$.\end{question}
        \begin{question} De wortels van de vergelijking $x^2+x+1=0$ zijn $x_{1,2}=\dfrac{-1\pm i\sqrt{3}}{2}$\end{question}
    \end{example}

    \begin{exercise} Bereken de wortels van volgende vergelijkingen.
        \begin{question} $x^2+2x+3=0$\end{question}
        \begin{question} $2x^2+3x+4=0$\end{question}
    \end{exercise}

    \begin{exercise} Ontbind volgende veeltermen in lineaire factoren.
        \begin{question} $x^2+2x+3$\end{question}
        \begin{question} $2x^2+3x+4$\end{question}
    \end{exercise}

\end{document}