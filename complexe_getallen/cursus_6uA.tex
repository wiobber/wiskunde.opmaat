\documentclass{xourse}
\input{../preamble.tex}
% \input{../preambles/tcolorbox.tex}   % voor formularia ...
\addPrintStyle{..}

\begin{document}
	\setcounter{tocdepth}{1}
    \xmtitle{Complexe getallen (6u)}{}  
    

\part{Complexe getallen}
\activitychapter{inleiding_vergelijkingen.tex}
\activitychapter{cmplx_definitie_algebraisch.tex}
%\activitychapter{cmplx_definitie_meetkundig.tex}
\activitychapter{cmplx_optellen_en_vermenigvuldigen.tex}
\activitychapter{cmplx_definitie_complexe_vlak.tex}  % na optellen en vermenigvuldigen...

\activitychapter{cmplx_norm.tex} 
\activitychapter{cmplx_complex_toegevoegde.tex}
\activitychapter{cmplx_inverse_en_deling.tex}
\activitychapter{cmplx_tweedegraadsvergelijkingen.tex}

% \practicesection{exercises/complexe_getallen_basis.tex}

\part{Goniometrische voorstelling}
%\part{Complexe getallen (8u)}
%\activitychapter{cmplx_vierkantswortels.tex}
%\activitychapter{cmplx_structuur.tex}


%\part{Complexe getallen (work-in-progress)}
%\activitychapter{cmplx_deling_dhz.tex}



% \part{Formularia}
% {\providecommand{\wraptitle}{Complexe getallen (versie Burgerlijk Ingenieur KU Leuven)}
% \activitychapter{../formularium/complexe_getallen.tex}
% }

\end{document}
