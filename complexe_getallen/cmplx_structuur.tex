\documentclass{ximera}
\input{../preamble}
\addPrintStyle{..}

\begin{document}
	\author{Wim Obbels}
	\xmtitle{De complexe getallen vormen een veld}{}
	\label{xim:cmplx_structuur}

    Wiskundigen hebben onderzocht waaraan dingen zouden moeten voldoen om ze redelijkerwijze 'getallen' te willen noemen. Het lijkt evident dat we 'getallen' willen kunnen optellen, aftrekken, vermenigvuldigen en delen. Van een voldoende rijke verzameling getallen verwachten we ook dat elk getal een tegengestelde heeft, en ook een inverse. Bovendien moeten de optelling en de vermenigvuldiging natuurlijk 'compatibel' zijn, zo willen we bijvoorbeeld dat $x+x = 2\cdot x$ voor elk getal $x$.

    Sinds het begin van de twintigste eeuw werden volgende eisen vastgelegd om voldoende eigenschappen van de 'gewone' reële getallen te behouden.
    We formuleren ze hier voor de complexe getallen $\C$.
    %We formuleren deze abstracte, maar eigenlijk evidente eigenschappen hier voor de complexe getallen $\C$.

    \begin{proposition} $\C,+$ is een commutatieve groep. Dat betekent

        \begin{tabular}{lrr@{ }llr}
            (Gs1) & $\forall u,v\in\C:$ & $u+v$&$\in\C$ & $\qquad\qquad$& (intern) \\
            (Gs2) & $\forall u,v,w\in\C:$ & $(u+v)+w = $&$u+(v+w)$  & $\qquad\qquad$& (associatief) \\
            (Gs3) & $\forall u\in\C:$ & $u+ 0 = $&$u = 0+u $  & $\qquad\qquad$& (neutraal element) \\
            (Gs4) & $\forall u\in\C,\exists -u\in\C:$ & $ u+ (-u) = $&$0 = (-u) +u $  & $\qquad\qquad$& (tegengestelde) \\
            (Gs5) & $\forall u,v \in\C:$ & $ u+ v = $&$v+ u $  & $\qquad\qquad$& (commutatief) 

        \end{tabular}
    \end{proposition}

    \begin{proposition} $\Cnul,\cdot$ is een commutatieve groep. Dat betekent

        \begin{tabular}{lr@{\qquad }r@{ }llr}
            (Gp1) & $\forall u,v\in\Cnul:$ & $u\cdot v$&$\in\C$ & $\qquad\qquad$& (intern) \\
            (Gp2) & $\forall u,v,w\in\Cnul:$ & $(u\cdot v)\cdot w = $&$u\cdot(v\cdot w)$  & $\qquad\qquad$& (associatief) \\
            (Gp3) & $\forall u\in\Cnul:$ & $u\cdot 1 = $&$u = 1\cdot u $  & $\qquad\qquad$& (neutraal element) \\
            (Gp4) & $\forall u\in\Cnul,\exists u^{-1}\in\C:$ & $ u\cdot u^{-1} =$&$ 1 = u^{-1}\cdot u $  & $\qquad\qquad$& (tegengestelde) \\
            (Gp5) & $\forall u,v \in\Cnul:$ & $ u \cdot v =$&$ v\cdot u $  & $\qquad\qquad$& (commutatief) 
        \end{tabular}
    \end{proposition}

    De compatibiliteit tussen plus en maal heet de \textbf{distributiviteit}:

    \begin{proposition} De vermenigvuldiging is distributief ten opzichte van de optelling:

    \begin{tabular}{lrr@{ }llr}
        (D1) & $\forall u,v,w \in\C:$ & $ u \cdot (v+w) = $&$ u\cdot v +u\cdot w$  & $\qquad\qquad$& ($\cdot$ is distributief t.o.v. $+$) 
    \end{tabular}
    \end{proposition}

    De voorgaande eigenschappen worden samengevat in het wiskundige begrip \textbf{veld}:

    \begin{definition} De complexe getallen $\C,+,\cdot$ met de optelling en de vermenigvuldiging vormen een \textbf{veld} omdat ze voldoen aan bovenstaande eigenschappen 
        van 
        
        \begin{tabular}{lll}
            1) & commutatieve groep voor de optelling  & (Gs1)-(Gs5),  \\
            2) & commutatieve groep voor de vermenigvuldiging  & (Gp1)-(Gp5), \\
            3) & distributiviteit van de optelling ten opzichte van de vermenigvuldiging  & (D1).
        \end{tabular}
    \end{definition}

    Ook de rationale getallen $\Q,+,\cdot$ en de reële getallen $\R,+,\cdot$ zijn voorbeelden van velden.

    \vspace{0.3cm} % HACK, to be done properly

    \begin{quickquestion*}{} Waarom zijn de natuurlijke getallen $\N,+,\cdot$ \textit{geen} veld? \end{quickquestion*}
    \begin{quickquestion*}{} Vormen de gehele getallen $\Z,+,\cdot$ een veld? \end{quickquestion*}

    \vspace{0.3cm} % HACK, to be done properly


    Een erg belangrijke en interessante eigenschap van reële getallen is bovendien dat ze geordend zijn:

    \begin{proposition} De reële getallen $\R,+,\cdot,\leq$ zijn een totaal geordend veld, dat betekent

        $\R,+,\cdot$ is een veld, en bovendien geldt

        \begin{tabular}{llllll}
            (Ot1) & $\forall x,y \in\R:$ & als $ x \leq y $ en  $y \leq x$ & dan $ x=y$       & $\qquad\qquad\qquad$& (antisymmetrie) \\
            (Ot2) & $\forall x,y,z \in\R:$ & als $ x \leq y $ en  $y \leq z$ & dan $x\leq z$  & $\qquad\qquad\qquad$& (transitief) \\
            (Ot3) & $\forall x,y \in\R:$ & $ u \leq v $ of  $v \leq u $ & (of beide)          & $\qquad\qquad\qquad$& (totaal) 
        \end{tabular}

        waarbij de orde $\leq$ compatibel is met som $+$ en het product $\cdot$, dus
        
        \begin{tabular}{llllll}
            (Oc1) & $\forall x,y,z \in\R:$ & als $ x \leq y$ & dan $x+z < y+z y$           & $\qquad$& (compatibel met $+$) \\
            (Oc2) & $\forall x,y \in\R:$ & als $ 0 \leq x$ en $0< y $& dan $0 < x\cdot y$  & $\qquad$& (compatibel met $\cdot$) 
        \end{tabular}

    \end{proposition}
         

\end{document}

